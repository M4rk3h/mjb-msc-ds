% Options for packages loaded elsewhere
\PassOptionsToPackage{unicode}{hyperref}
\PassOptionsToPackage{hyphens}{url}
%
\documentclass[
]{article}
\usepackage{amsmath,amssymb}
\usepackage{lmodern}
\usepackage{ifxetex,ifluatex}
\ifnum 0\ifxetex 1\fi\ifluatex 1\fi=0 % if pdftex
  \usepackage[T1]{fontenc}
  \usepackage[utf8]{inputenc}
  \usepackage{textcomp} % provide euro and other symbols
\else % if luatex or xetex
  \usepackage{unicode-math}
  \defaultfontfeatures{Scale=MatchLowercase}
  \defaultfontfeatures[\rmfamily]{Ligatures=TeX,Scale=1}
\fi
% Use upquote if available, for straight quotes in verbatim environments
\IfFileExists{upquote.sty}{\usepackage{upquote}}{}
\IfFileExists{microtype.sty}{% use microtype if available
  \usepackage[]{microtype}
  \UseMicrotypeSet[protrusion]{basicmath} % disable protrusion for tt fonts
}{}
\makeatletter
\@ifundefined{KOMAClassName}{% if non-KOMA class
  \IfFileExists{parskip.sty}{%
    \usepackage{parskip}
  }{% else
    \setlength{\parindent}{0pt}
    \setlength{\parskip}{6pt plus 2pt minus 1pt}}
}{% if KOMA class
  \KOMAoptions{parskip=half}}
\makeatother
\usepackage{xcolor}
\IfFileExists{xurl.sty}{\usepackage{xurl}}{} % add URL line breaks if available
\IfFileExists{bookmark.sty}{\usepackage{bookmark}}{\usepackage{hyperref}}
\hypersetup{
  pdftitle={MS4S09 Coursework 2 - 2020/21},
  pdfauthor={Mark Baber},
  hidelinks,
  pdfcreator={LaTeX via pandoc}}
\urlstyle{same} % disable monospaced font for URLs
\usepackage[margin=1in]{geometry}
\usepackage{color}
\usepackage{fancyvrb}
\newcommand{\VerbBar}{|}
\newcommand{\VERB}{\Verb[commandchars=\\\{\}]}
\DefineVerbatimEnvironment{Highlighting}{Verbatim}{commandchars=\\\{\}}
% Add ',fontsize=\small' for more characters per line
\usepackage{framed}
\definecolor{shadecolor}{RGB}{248,248,248}
\newenvironment{Shaded}{\begin{snugshade}}{\end{snugshade}}
\newcommand{\AlertTok}[1]{\textcolor[rgb]{0.94,0.16,0.16}{#1}}
\newcommand{\AnnotationTok}[1]{\textcolor[rgb]{0.56,0.35,0.01}{\textbf{\textit{#1}}}}
\newcommand{\AttributeTok}[1]{\textcolor[rgb]{0.77,0.63,0.00}{#1}}
\newcommand{\BaseNTok}[1]{\textcolor[rgb]{0.00,0.00,0.81}{#1}}
\newcommand{\BuiltInTok}[1]{#1}
\newcommand{\CharTok}[1]{\textcolor[rgb]{0.31,0.60,0.02}{#1}}
\newcommand{\CommentTok}[1]{\textcolor[rgb]{0.56,0.35,0.01}{\textit{#1}}}
\newcommand{\CommentVarTok}[1]{\textcolor[rgb]{0.56,0.35,0.01}{\textbf{\textit{#1}}}}
\newcommand{\ConstantTok}[1]{\textcolor[rgb]{0.00,0.00,0.00}{#1}}
\newcommand{\ControlFlowTok}[1]{\textcolor[rgb]{0.13,0.29,0.53}{\textbf{#1}}}
\newcommand{\DataTypeTok}[1]{\textcolor[rgb]{0.13,0.29,0.53}{#1}}
\newcommand{\DecValTok}[1]{\textcolor[rgb]{0.00,0.00,0.81}{#1}}
\newcommand{\DocumentationTok}[1]{\textcolor[rgb]{0.56,0.35,0.01}{\textbf{\textit{#1}}}}
\newcommand{\ErrorTok}[1]{\textcolor[rgb]{0.64,0.00,0.00}{\textbf{#1}}}
\newcommand{\ExtensionTok}[1]{#1}
\newcommand{\FloatTok}[1]{\textcolor[rgb]{0.00,0.00,0.81}{#1}}
\newcommand{\FunctionTok}[1]{\textcolor[rgb]{0.00,0.00,0.00}{#1}}
\newcommand{\ImportTok}[1]{#1}
\newcommand{\InformationTok}[1]{\textcolor[rgb]{0.56,0.35,0.01}{\textbf{\textit{#1}}}}
\newcommand{\KeywordTok}[1]{\textcolor[rgb]{0.13,0.29,0.53}{\textbf{#1}}}
\newcommand{\NormalTok}[1]{#1}
\newcommand{\OperatorTok}[1]{\textcolor[rgb]{0.81,0.36,0.00}{\textbf{#1}}}
\newcommand{\OtherTok}[1]{\textcolor[rgb]{0.56,0.35,0.01}{#1}}
\newcommand{\PreprocessorTok}[1]{\textcolor[rgb]{0.56,0.35,0.01}{\textit{#1}}}
\newcommand{\RegionMarkerTok}[1]{#1}
\newcommand{\SpecialCharTok}[1]{\textcolor[rgb]{0.00,0.00,0.00}{#1}}
\newcommand{\SpecialStringTok}[1]{\textcolor[rgb]{0.31,0.60,0.02}{#1}}
\newcommand{\StringTok}[1]{\textcolor[rgb]{0.31,0.60,0.02}{#1}}
\newcommand{\VariableTok}[1]{\textcolor[rgb]{0.00,0.00,0.00}{#1}}
\newcommand{\VerbatimStringTok}[1]{\textcolor[rgb]{0.31,0.60,0.02}{#1}}
\newcommand{\WarningTok}[1]{\textcolor[rgb]{0.56,0.35,0.01}{\textbf{\textit{#1}}}}
\usepackage{longtable,booktabs,array}
\usepackage{calc} % for calculating minipage widths
% Correct order of tables after \paragraph or \subparagraph
\usepackage{etoolbox}
\makeatletter
\patchcmd\longtable{\par}{\if@noskipsec\mbox{}\fi\par}{}{}
\makeatother
% Allow footnotes in longtable head/foot
\IfFileExists{footnotehyper.sty}{\usepackage{footnotehyper}}{\usepackage{footnote}}
\makesavenoteenv{longtable}
\usepackage{graphicx}
\makeatletter
\def\maxwidth{\ifdim\Gin@nat@width>\linewidth\linewidth\else\Gin@nat@width\fi}
\def\maxheight{\ifdim\Gin@nat@height>\textheight\textheight\else\Gin@nat@height\fi}
\makeatother
% Scale images if necessary, so that they will not overflow the page
% margins by default, and it is still possible to overwrite the defaults
% using explicit options in \includegraphics[width, height, ...]{}
\setkeys{Gin}{width=\maxwidth,height=\maxheight,keepaspectratio}
% Set default figure placement to htbp
\makeatletter
\def\fps@figure{htbp}
\makeatother
\setlength{\emergencystretch}{3em} % prevent overfull lines
\providecommand{\tightlist}{%
  \setlength{\itemsep}{0pt}\setlength{\parskip}{0pt}}
\setcounter{secnumdepth}{-\maxdimen} % remove section numbering
\ifluatex
  \usepackage{selnolig}  % disable illegal ligatures
\fi

\title{MS4S09 Coursework 2 - 2020/21}
\usepackage{etoolbox}
\makeatletter
\providecommand{\subtitle}[1]{% add subtitle to \maketitle
  \apptocmd{\@title}{\par {\large #1 \par}}{}{}
}
\makeatother
\subtitle{17076749}
\author{Mark Baber}
\date{xx/xx/2021}

\begin{document}
\maketitle

This report will look at using data mining techniques on a time series
dataset. This report will be broken down into several parts, from
getting the data, exploring the data, looking into trend and seasonality
of the data before continuing onto a more indepth analysis. This indepth
analysis will cover ARMA and forecasting.

All of this will be done using R and Rstudio, with the use of limited
packages which has been added below: - magrittr - tseries - knitr

\hypertarget{task-1-getting-the-data-10}{%
\section{1 Task 1 -- Getting the data
(10\%)}\label{task-1-getting-the-data-10}}

Write an R script that downloads the data directly from the website for
the 30 time series (3 time series for each of the 10 districts) using
the ``Year ordered statistics'' option, and selecting the districts
listed. Download up to December 2021.

Create the 30 time-series objects in R to store the data you have
downloaded. Remember to specify the appropriate starting point and
frequency.

\begin{Shaded}
\begin{Highlighting}[]
\CommentTok{\# set address}
\NormalTok{address }\OtherTok{\textless{}{-}} \StringTok{"https://www.metoffice.gov.uk/pub/data/weather/uk/climate/datasets/"}
\CommentTok{\# set features}
\NormalTok{features }\OtherTok{\textless{}{-}} \FunctionTok{c}\NormalTok{(}\StringTok{"Tmax"}\NormalTok{, }\StringTok{"Tmean"}\NormalTok{, }\StringTok{"Tmin"}\NormalTok{)}
\CommentTok{\# create a list of districts}
\NormalTok{districts }\OtherTok{\textless{}{-}} \FunctionTok{c}\NormalTok{(}\StringTok{"Northern\_Ireland"}\NormalTok{,}
                  \StringTok{"Scotland\_N"}\NormalTok{,}
                  \StringTok{"Scotland\_E"}\NormalTok{,}
                  \StringTok{"Scotland\_W"}\NormalTok{,}
                  \StringTok{"England\_E\_and\_NE"}\NormalTok{,}
                  \StringTok{"England\_NW\_and\_N\_Wales"}\NormalTok{,}
                  \StringTok{"Midlands"}\NormalTok{,}
                  \StringTok{"East\_Anglia"}\NormalTok{,}
                  \StringTok{"England\_SW\_and\_S\_Wales"}\NormalTok{,}
                  \StringTok{"England\_SE\_and\_Central\_S"}\NormalTok{)}
\end{Highlighting}
\end{Shaded}

The section above looked to set up the base url for the datasets, the 3
different features which changed within the url and the 10 districts
which will be needed for the url. Before going further into creating a
function, there was some pre-analysis on the scraped dataset to
calculate the number of rows and looked into which rows \& columns
should be omitted.

The next step would be to create the function which will grab each
dataset, from the url using the base address, features and all
districts.

Now that the function has managed to get all datasets for each feature
(TMAX-TMEAN-TMIN) lets move on to the next step which will be to explore
these datasets.

\hypertarget{task-2-r-programming-5}{%
\section{2 - Task 2 -- R programming
(5\%)}\label{task-2-r-programming-5}}

Write an R script to identify the district and date (year and month) of
the highest and the lowest max, min and mean temperature (six results in
total).

This section of the report will look to calculate the max, mean and min
tempertures for each subset of data whilst pointing out the district and
date.

\begin{Shaded}
\begin{Highlighting}[]
\CommentTok{\# 2 {-} Task 2}
\CommentTok{\# find the max value\textquotesingle{}s index}
\NormalTok{maxIndex }\OtherTok{\textless{}{-}}\NormalTok{ Data}\SpecialCharTok{$}\NormalTok{Tmax }\SpecialCharTok{\%\textgreater{}\%} 
  \FunctionTok{unlist}\NormalTok{() }\SpecialCharTok{\%\textgreater{}\%} 
  \FunctionTok{as.vector}\NormalTok{() }\SpecialCharTok{\%\textgreater{}\%} 
  \FunctionTok{which.max}\NormalTok{() }

\CommentTok{\# find sub position}
\NormalTok{subIndex }\OtherTok{\textless{}{-}} \FunctionTok{round}\NormalTok{((maxIndex}\SpecialCharTok{/}\DecValTok{16440}\NormalTok{)}\SpecialCharTok{*}\DecValTok{10}\NormalTok{) }\CommentTok{\# 8 {-} This could be district}

\CommentTok{\# find max value}
\CommentTok{\# maxVal \textless{}{-} DTmUn \%\textgreater{}\% which.max()}

\CommentTok{\# find year}
\CommentTok{\# maxYear \textless{}{-} floor(time(DTmUn))[which.max(DTmUn)]}

\CommentTok{\# month}
\CommentTok{\# maxMonth \textless{}{-} month.abb[(time(Data$Tmax)[which.max(Data$Tmax)] \%\textgreater{}\% 1)*12+1]}

\CommentTok{\# calculate regions}
\CommentTok{\# maxRegions \textless{}{-} names(Data$Tmax)}

\CommentTok{\# GETS MONTH}
\NormalTok{unlistedTMAX }\OtherTok{\textless{}{-}}\NormalTok{ Data}\SpecialCharTok{$}\NormalTok{Tmax }\SpecialCharTok{\%\textgreater{}\%} \FunctionTok{unlist}\NormalTok{()}
\NormalTok{myMm }\OtherTok{\textless{}{-}}\NormalTok{ month.abb[(}\FunctionTok{time}\NormalTok{(unlistedTMAX)[}\FunctionTok{which.min}\NormalTok{(unlistedTMAX)] }\SpecialCharTok{\%\%} \DecValTok{1}\NormalTok{)}\SpecialCharTok{*}\DecValTok{12}\SpecialCharTok{+}\DecValTok{1}\NormalTok{]}
\CommentTok{\# plot(Data$Tmax$Northern\_Ireland, type = \textquotesingle{}l\textquotesingle{})}

\CommentTok{\# get max temp of Tmax}
\FunctionTok{sapply}\NormalTok{(Data}\SpecialCharTok{$}\NormalTok{Tmax, which.max)}
\end{Highlighting}
\end{Shaded}

\begin{verbatim}
##         Northern_Ireland               Scotland_N               Scotland_E 
##                     1340                      764                     1471 
##               Scotland_W         England_E_and_NE   England_NW_and_N_Wales 
##                      764                     1471                     1471 
##                 Midlands              East_Anglia   England_SW_and_S_Wales 
##                     1471                     1471                     1340 
## England_SE_and_Central_S 
##                     1471
\end{verbatim}

\begin{Shaded}
\begin{Highlighting}[]
\CommentTok{\# data\_min\_value\_time \textless{}{-} time(Data)[which.min(Data)]}

\CommentTok{\# unlist tMAX}
\NormalTok{tmax\_unl }\OtherTok{\textless{}{-}}\NormalTok{ Data}\SpecialCharTok{$}\NormalTok{Tmax }\SpecialCharTok{\%\textgreater{}\%} 
  \FunctionTok{unlist}\NormalTok{() }\SpecialCharTok{\%\textgreater{}\%} 
  \FunctionTok{as.vector}\NormalTok{()}
\end{Highlighting}
\end{Shaded}

For task 2, I wasn't able to figure this out so this section will be
left out.

\hypertarget{task-3-exploratory-data-analysis-25}{%
\section{3 - Task 3 -- Exploratory Data Analysis
(25\%)}\label{task-3-exploratory-data-analysis-25}}

Carry out an EDA of the data you have downloaded. In order to complete
your analysis, you may find it useful to answer (but not only!) the
following questions:

− Which district is the coldest/warmest? Describe used criteria. − Which
district has the widest temperature range? − Are winters/summers getting
colder/hotter?

\begin{Shaded}
\begin{Highlighting}[]
\CommentTok{\# find max value}
\NormalTok{colMax }\OtherTok{\textless{}{-}} \ControlFlowTok{function}\NormalTok{(data) }\FunctionTok{sapply}\NormalTok{(data, max, }\AttributeTok{na.rm =} \ConstantTok{TRUE}\NormalTok{)}
\FunctionTok{colMax}\NormalTok{(Data}\SpecialCharTok{$}\NormalTok{Tmax) }\SpecialCharTok{\%\textgreater{}\%} \FunctionTok{which.max}\NormalTok{()}
\end{Highlighting}
\end{Shaded}

\begin{verbatim}
## East_Anglia 
##           8
\end{verbatim}

\begin{Shaded}
\begin{Highlighting}[]
\CommentTok{\# East\_Anglia has the highest temp within the Tmax series.}
\FunctionTok{colMax}\NormalTok{(Data}\SpecialCharTok{$}\NormalTok{Tmean) }\SpecialCharTok{\%\textgreater{}\%} \FunctionTok{which.max}\NormalTok{()}
\end{Highlighting}
\end{Shaded}

\begin{verbatim}
## East_Anglia 
##           8
\end{verbatim}

\begin{Shaded}
\begin{Highlighting}[]
\CommentTok{\# East\_Anglia has the highest temp within the Tmean series.}
\FunctionTok{colMax}\NormalTok{(Data}\SpecialCharTok{$}\NormalTok{Tmin) }\SpecialCharTok{\%\textgreater{}\%} \FunctionTok{which.max}\NormalTok{()}
\end{Highlighting}
\end{Shaded}

\begin{verbatim}
## England_SE_and_Central_S 
##                       10
\end{verbatim}

\begin{Shaded}
\begin{Highlighting}[]
\CommentTok{\# England\_SE\_and\_Central\_S has the highest temp within Tmin.}
\end{Highlighting}
\end{Shaded}

Above I have looked at the max temperatures whilst also finding the max
of those with which.max. Now lets create a function to find the lowest
temperatures.

\begin{Shaded}
\begin{Highlighting}[]
\CommentTok{\# find lowest value}
\NormalTok{colMin }\OtherTok{\textless{}{-}} \ControlFlowTok{function}\NormalTok{(data) }\FunctionTok{sapply}\NormalTok{(data, min, }\AttributeTok{na.rm =} \ConstantTok{TRUE}\NormalTok{)}
\FunctionTok{colMin}\NormalTok{(Data}\SpecialCharTok{$}\NormalTok{Tmax) }\SpecialCharTok{\%\textgreater{}\%} \FunctionTok{which.min}\NormalTok{()}
\end{Highlighting}
\end{Shaded}

\begin{verbatim}
## Midlands 
##        7
\end{verbatim}

\begin{Shaded}
\begin{Highlighting}[]
\CommentTok{\# Midlands has the lowest temp within the Tmax series.}
\FunctionTok{colMin}\NormalTok{(Data}\SpecialCharTok{$}\NormalTok{Tmean) }\SpecialCharTok{\%\textgreater{}\%} \FunctionTok{which.min}\NormalTok{()}
\end{Highlighting}
\end{Shaded}

\begin{verbatim}
## Scotland_E 
##          3
\end{verbatim}

\begin{Shaded}
\begin{Highlighting}[]
\CommentTok{\# Scotland\_E has the lowest temp within the Tmean series.}
\FunctionTok{colMin}\NormalTok{(Data}\SpecialCharTok{$}\NormalTok{Tmin) }\SpecialCharTok{\%\textgreater{}\%} \FunctionTok{which.min}\NormalTok{()}
\end{Highlighting}
\end{Shaded}

\begin{verbatim}
## Scotland_E 
##          3
\end{verbatim}

\begin{Shaded}
\begin{Highlighting}[]
\CommentTok{\# Scotland\_E has the lowest temp within the Tmin series}
\end{Highlighting}
\end{Shaded}

This section looked at the lowest temperatures within the dataset, with
Midlands having the lowest temp within Tmax, Scotland\_E having the
lowest within Tmean and again Scotland\_E having the lowest for Tmin.

\begin{Shaded}
\begin{Highlighting}[]
\NormalTok{colRange }\OtherTok{\textless{}{-}} \ControlFlowTok{function}\NormalTok{(data) }\FunctionTok{sapply}\NormalTok{(data, range, }\AttributeTok{na.rm =} \ConstantTok{TRUE}\NormalTok{)}
\FunctionTok{colRange}\NormalTok{(Data}\SpecialCharTok{$}\NormalTok{Tmax)}
\end{Highlighting}
\end{Shaded}

\begin{verbatim}
##      Northern_Ireland Scotland_N Scotland_E Scotland_W England_E_and_NE
## [1,]              1.5        0.4       -0.5        0.6             -0.1
## [2,]             22.1       20.1       21.4       21.6             24.4
##      England_NW_and_N_Wales Midlands East_Anglia England_SW_and_S_Wales
## [1,]                   -0.2     -0.6        -0.2                    0.3
## [2,]                   23.3     25.7        26.7                   24.3
##      England_SE_and_Central_S
## [1,]                     -0.1
## [2,]                     26.1
\end{verbatim}

\begin{Shaded}
\begin{Highlighting}[]
\FunctionTok{colRange}\NormalTok{(Data}\SpecialCharTok{$}\NormalTok{Tmean)}
\end{Highlighting}
\end{Shaded}

\begin{verbatim}
##      Northern_Ireland Scotland_N Scotland_E Scotland_W England_E_and_NE
## [1,]             -0.7       -2.4       -3.5       -2.3             -2.0
## [2,]             17.0       15.0       16.0       16.1             18.3
##      England_NW_and_N_Wales Midlands East_Anglia England_SW_and_S_Wales
## [1,]                   -2.6     -2.8        -2.5                   -2.4
## [2,]                   17.9     19.5        20.4                   18.8
##      England_SE_and_Central_S
## [1,]                     -2.7
## [2,]                     20.2
\end{verbatim}

\begin{Shaded}
\begin{Highlighting}[]
\FunctionTok{colRange}\NormalTok{(Data}\SpecialCharTok{$}\NormalTok{Tmin)}
\end{Highlighting}
\end{Shaded}

\begin{verbatim}
##      Northern_Ireland Scotland_N Scotland_E Scotland_W England_E_and_NE
## [1,]             -4.3       -6.3       -7.4       -5.6             -5.5
## [2,]             12.2       11.0       11.0       11.6             12.6
##      England_NW_and_N_Wales Midlands East_Anglia England_SW_and_S_Wales
## [1,]                   -5.8     -6.6        -5.9                   -5.4
## [2,]                   12.7     13.5        14.6                   13.9
##      England_SE_and_Central_S
## [1,]                     -5.8
## [2,]                     14.7
\end{verbatim}

This section looked at the ranges of all the dataset features, all of
these could also be done with a couple of functions on the whole
dataset.

\begin{Shaded}
\begin{Highlighting}[]
\CommentTok{\# The above can also be done on the full dataset.}
\NormalTok{maxTemps }\OtherTok{\textless{}{-}} \FunctionTok{sapply}\NormalTok{(Data, colMax)}
\CommentTok{\# print}
\NormalTok{maxTemps}
\end{Highlighting}
\end{Shaded}

\begin{verbatim}
##                          Tmax Tmean Tmin
## Northern_Ireland         22.1  17.0 12.2
## Scotland_N               20.1  15.0 11.0
## Scotland_E               21.4  16.0 11.0
## Scotland_W               21.6  16.1 11.6
## England_E_and_NE         24.4  18.3 12.6
## England_NW_and_N_Wales   23.3  17.9 12.7
## Midlands                 25.7  19.5 13.5
## East_Anglia              26.7  20.4 14.6
## England_SW_and_S_Wales   24.3  18.8 13.9
## England_SE_and_Central_S 26.1  20.2 14.7
\end{verbatim}

\begin{Shaded}
\begin{Highlighting}[]
\NormalTok{minTemps }\OtherTok{\textless{}{-}} \FunctionTok{sapply}\NormalTok{(Data, colMin)}
\CommentTok{\# print}
\NormalTok{minTemps}
\end{Highlighting}
\end{Shaded}

\begin{verbatim}
##                          Tmax Tmean Tmin
## Northern_Ireland          1.5  -0.7 -4.3
## Scotland_N                0.4  -2.4 -6.3
## Scotland_E               -0.5  -3.5 -7.4
## Scotland_W                0.6  -2.3 -5.6
## England_E_and_NE         -0.1  -2.0 -5.5
## England_NW_and_N_Wales   -0.2  -2.6 -5.8
## Midlands                 -0.6  -2.8 -6.6
## East_Anglia              -0.2  -2.5 -5.9
## England_SW_and_S_Wales    0.3  -2.4 -5.4
## England_SE_and_Central_S -0.1  -2.7 -5.8
\end{verbatim}

\begin{Shaded}
\begin{Highlighting}[]
\NormalTok{totRange }\OtherTok{\textless{}{-}} \FunctionTok{sapply}\NormalTok{(Data, colRange)}
\CommentTok{\# print}
\NormalTok{totRange}
\end{Highlighting}
\end{Shaded}

\begin{verbatim}
##       Tmax Tmean Tmin
##  [1,]  1.5  -0.7 -4.3
##  [2,] 22.1  17.0 12.2
##  [3,]  0.4  -2.4 -6.3
##  [4,] 20.1  15.0 11.0
##  [5,] -0.5  -3.5 -7.4
##  [6,] 21.4  16.0 11.0
##  [7,]  0.6  -2.3 -5.6
##  [8,] 21.6  16.1 11.6
##  [9,] -0.1  -2.0 -5.5
## [10,] 24.4  18.3 12.6
## [11,] -0.2  -2.6 -5.8
## [12,] 23.3  17.9 12.7
## [13,] -0.6  -2.8 -6.6
## [14,] 25.7  19.5 13.5
## [15,] -0.2  -2.5 -5.9
## [16,] 26.7  20.4 14.6
## [17,]  0.3  -2.4 -5.4
## [18,] 24.3  18.8 13.9
## [19,] -0.1  -2.7 -5.8
## [20,] 26.1  20.2 14.7
\end{verbatim}

\begin{Shaded}
\begin{Highlighting}[]
\NormalTok{colAvg }\OtherTok{\textless{}{-}} \ControlFlowTok{function}\NormalTok{(data) }\FunctionTok{sapply}\NormalTok{(data, mean)}
\FunctionTok{sapply}\NormalTok{(Data, colAvg)}
\end{Highlighting}
\end{Shaded}

\begin{verbatim}
##                              Tmax    Tmean     Tmin
## Northern_Ireland         12.09057 8.571959 5.070985
## Scotland_N               10.03461 6.844161 3.740511
## Scotland_E               10.47251 6.894161 3.373601
## Scotland_W               11.03163 7.687348 4.436557
## England_E_and_NE         12.16679 8.412713 4.691423
## England_NW_and_N_Wales   11.91837 8.468187 5.057482
## Midlands                 12.94009 8.990815 5.052798
## East_Anglia              13.69738 9.621594 5.558698
## England_SW_and_S_Wales   12.99221 9.424270 5.877251
## England_SE_and_Central_S 13.82950 9.759793 5.711010
\end{verbatim}

\hypertarget{task-4-trend-and-seasonality}{%
\section{4 - Task 4 -- Trend and
Seasonality}\label{task-4-trend-and-seasonality}}

\hypertarget{subset}{%
\section{4.0 - Subset}\label{subset}}

For each district, consider the 3 time series: max, mean, min. subset
each of the 30 time series until December 2019.

This section plans to subset the dataset for the time series datasets.

\begin{Shaded}
\begin{Highlighting}[]
\DocumentationTok{\#\# NONE OF THIS WORKS}
\CommentTok{\# Time Series function}
\NormalTok{convert.ts }\OtherTok{\textless{}{-}} \ControlFlowTok{function}\NormalTok{(data, feature, district)\{ }\CommentTok{\# pass 3 parameters }
  \FunctionTok{c}\NormalTok{(data, feature, districts) }\SpecialCharTok{\%\textgreater{}\%}  
    \FunctionTok{paste}\NormalTok{(}\AttributeTok{collapse =} \StringTok{"$"}\NormalTok{) }\CommentTok{\# add $ between each parameter}
\NormalTok{\}}

\NormalTok{matrix.ts }\OtherTok{\textless{}{-}} \ControlFlowTok{function}\NormalTok{(feature)\{}
  \FunctionTok{lapply}\NormalTok{(data, convert.ts, }\AttributeTok{feature =}\NormalTok{ feature, }\AttributeTok{district =}\NormalTok{ districts)}
\NormalTok{\}}

\NormalTok{tsMatrix }\OtherTok{\textless{}{-}} \FunctionTok{matrix.ts}\NormalTok{(Data)}

\NormalTok{matCon }\OtherTok{\textless{}{-}} \ControlFlowTok{function}\NormalTok{(data)\{}
\NormalTok{  data }\SpecialCharTok{\%\textgreater{}\%} 
    \FunctionTok{t}\NormalTok{() }\SpecialCharTok{\%\textgreater{}\%} 
    \FunctionTok{as.vector}\NormalTok{()}
\NormalTok{\}}
\NormalTok{testTS }\OtherTok{\textless{}{-}} \FunctionTok{lapply}\NormalTok{(Data, matCon)}
\FunctionTok{is.vector}\NormalTok{(testTS)}
\end{Highlighting}
\end{Shaded}

\begin{verbatim}
## [1] TRUE
\end{verbatim}

\begin{Shaded}
\begin{Highlighting}[]
\CommentTok{\# \textasciitilde{}\textasciitilde{}\textasciitilde{}\textasciitilde{}\textasciitilde{}\textasciitilde{}\textasciitilde{}\textasciitilde{}\textasciitilde{}\textasciitilde{}\textasciitilde{}\textasciitilde{}\textasciitilde{}\textasciitilde{}\textasciitilde{}\textasciitilde{}\textasciitilde{}\textasciitilde{}\textasciitilde{}\textasciitilde{}\textasciitilde{}\textasciitilde{}\textasciitilde{}\textasciitilde{}\textasciitilde{}\textasciitilde{}\textasciitilde{}\textasciitilde{}\textasciitilde{}\textasciitilde{}\textasciitilde{}\textasciitilde{}\textasciitilde{}\textasciitilde{}\textasciitilde{}\textasciitilde{}\textasciitilde{}\textasciitilde{}\textasciitilde{}\textasciitilde{}\textasciitilde{}\textasciitilde{}\textasciitilde{}\textasciitilde{}\textasciitilde{}\textasciitilde{}\textasciitilde{}\textasciitilde{}\textasciitilde{}\textasciitilde{}\textasciitilde{}\textasciitilde{}\textasciitilde{}\textasciitilde{}\textasciitilde{}\textasciitilde{}\textasciitilde{}\textasciitilde{}\textasciitilde{}\textasciitilde{}\textasciitilde{}\textasciitilde{}\textasciitilde{}\textasciitilde{}\textasciitilde{}\textasciitilde{}\textasciitilde{}\textasciitilde{}\textasciitilde{}\textasciitilde{}\textasciitilde{}\textasciitilde{}\textasciitilde{}\textasciitilde{}\textasciitilde{}\textasciitilde{}}

\CommentTok{\# function to slice dataset}
\NormalTok{slice.ts }\OtherTok{\textless{}{-}} \ControlFlowTok{function}\NormalTok{(data)\{}
\NormalTok{  data }\SpecialCharTok{\%\textgreater{}\%} \FunctionTok{window}\NormalTok{(}\AttributeTok{start =} \FunctionTok{c}\NormalTok{(}\DecValTok{1884}\NormalTok{,}\DecValTok{12}\NormalTok{), }
           \AttributeTok{end =} \FunctionTok{c}\NormalTok{(}\DecValTok{2019}\NormalTok{,}\DecValTok{12}\NormalTok{), }
           \AttributeTok{frequency =} \DecValTok{12}\NormalTok{,}
           \AttributeTok{extend =} \ConstantTok{FALSE}\NormalTok{)}
\NormalTok{\}}
\CommentTok{\# use fun for each dataset.}
\NormalTok{slicedWindow }\OtherTok{\textless{}{-}} 
  \FunctionTok{slice.ts}\NormalTok{(Data}\SpecialCharTok{$}\NormalTok{Tmax}\SpecialCharTok{$}\NormalTok{England\_SE\_and\_Central\_S)}
\end{Highlighting}
\end{Shaded}

The section above managed to slice a dataset at a time, but struggled to
do this for all of the datasets within `Data', I struggled converting
the list to a matrix for each section.

\begin{Shaded}
\begin{Highlighting}[]
\CommentTok{\# 4.1 {-} Estimate Trend}
\CommentTok{\# Estimate the trend of each time series using linear, quadratic and cubic }
\CommentTok{\# regression. Compare your results and use appropriate plots and/or tables }
\CommentTok{\# to confirm your observations.}

\CommentTok{\# \textasciitilde{}\textasciitilde{}\textasciitilde{}\textasciitilde{}\textasciitilde{}\textasciitilde{}\textasciitilde{}\textasciitilde{}\textasciitilde{}\textasciitilde{}\textasciitilde{}\textasciitilde{}\textasciitilde{}\textasciitilde{}\textasciitilde{}\textasciitilde{}\textasciitilde{}\textasciitilde{}\textasciitilde{}\textasciitilde{}\textasciitilde{}\textasciitilde{}\textasciitilde{}\textasciitilde{}\textasciitilde{}\textasciitilde{}\textasciitilde{}\textasciitilde{}\textasciitilde{}\textasciitilde{}\textasciitilde{}\textasciitilde{}\textasciitilde{}\textasciitilde{}\textasciitilde{}\textasciitilde{}\textasciitilde{}\textasciitilde{}\textasciitilde{}\textasciitilde{}\textasciitilde{}\textasciitilde{}\textasciitilde{}\textasciitilde{}\textasciitilde{}\textasciitilde{}\textasciitilde{}\textasciitilde{}\textasciitilde{}\textasciitilde{}\textasciitilde{}\textasciitilde{}\textasciitilde{}\textasciitilde{}\textasciitilde{}\textasciitilde{}\textasciitilde{}\textasciitilde{}\textasciitilde{}\textasciitilde{}\textasciitilde{}\textasciitilde{}\textasciitilde{}\textasciitilde{}\textasciitilde{}\textasciitilde{}\textasciitilde{}\textasciitilde{}\textasciitilde{}\textasciitilde{}\textasciitilde{}\textasciitilde{}\textasciitilde{}\textasciitilde{}\textasciitilde{}\textasciitilde{}\textasciitilde{}}
\CommentTok{\# create time vector}
\CommentTok{\# create a time vector}
\NormalTok{time }\OtherTok{\textless{}{-}} \DecValTok{1}\SpecialCharTok{:}\FunctionTok{length}\NormalTok{(slicedWindow)}
\CommentTok{\# re{-}scale from 0 {-} 1}
\NormalTok{time }\OtherTok{\textless{}{-}}\NormalTok{ (time }\SpecialCharTok{{-}} \FunctionTok{min}\NormalTok{(time))}\SpecialCharTok{/}\NormalTok{(}\FunctionTok{max}\NormalTok{(time) }\SpecialCharTok{{-}} \DecValTok{1}\NormalTok{)}
\CommentTok{\# function for linear model}
\NormalTok{linear.fun }\OtherTok{\textless{}{-}} \ControlFlowTok{function}\NormalTok{(timeseries)\{}
  \CommentTok{\# create linear trend}
\NormalTok{  linear.fit }\OtherTok{\textless{}\textless{}{-}} \FunctionTok{lm}\NormalTok{(timeseries }\SpecialCharTok{\textasciitilde{}}\NormalTok{ time) }\CommentTok{\# use \textless{}\textless{}{-} for global variables}
    \CommentTok{\# create linear fitted}
\NormalTok{  linear.fit }\SpecialCharTok{\%\textgreater{}\%} \FunctionTok{fitted}\NormalTok{() }\SpecialCharTok{\%\textgreater{}\%} \FunctionTok{ts}\NormalTok{(}\AttributeTok{start =} \FunctionTok{c}\NormalTok{(}\DecValTok{1884}\NormalTok{,}\DecValTok{12}\NormalTok{), }\AttributeTok{end =} \FunctionTok{c}\NormalTok{(}\DecValTok{2019}\NormalTok{,}\DecValTok{12}\NormalTok{), }\AttributeTok{frequency =} \DecValTok{12}\NormalTok{) }\OtherTok{{-}\textgreater{}\textgreater{}}\NormalTok{ linear.fitted}
\NormalTok{\}}
\CommentTok{\# run linear fun}
\FunctionTok{linear.fun}\NormalTok{(slicedWindow)}
\CommentTok{\# check summary}
\FunctionTok{summary}\NormalTok{(linear.fit)}
\end{Highlighting}
\end{Shaded}

\begin{verbatim}
## 
## Call:
## lm(formula = timeseries ~ time)
## 
## Residuals:
##      Min       1Q   Median       3Q      Max 
## -14.0220  -4.8619  -0.1373   5.0316  11.9171 
## 
## Coefficients:
##             Estimate Std. Error t value Pr(>|t|)    
## (Intercept)  13.1047     0.2736  47.900   <2e-16 ***
## time          1.4131     0.4738   2.983   0.0029 ** 
## ---
## Signif. codes:  0 '***' 0.001 '**' 0.01 '*' 0.05 '.' 0.1 ' ' 1
## 
## Residual standard error: 5.51 on 1619 degrees of freedom
## Multiple R-squared:  0.005465,   Adjusted R-squared:  0.00485 
## F-statistic: 8.896 on 1 and 1619 DF,  p-value: 0.002901
\end{verbatim}

\begin{Shaded}
\begin{Highlighting}[]
\CommentTok{\# Now try to plot}
\FunctionTok{ts.plot}\NormalTok{(slicedWindow, }\AttributeTok{ylab =} \StringTok{"Temperature"}\NormalTok{)}
\CommentTok{\# add linear fitted lines}
\FunctionTok{lines}\NormalTok{(linear.fitted, }\AttributeTok{col =} \StringTok{"green"}\NormalTok{, }\AttributeTok{lwd =} \DecValTok{2}\NormalTok{)}
\CommentTok{\# add mean line}
\FunctionTok{abline}\NormalTok{(}\FunctionTok{mean}\NormalTok{(slicedWindow), }\DecValTok{0}\NormalTok{, }\AttributeTok{col =} \StringTok{"blue"}\NormalTok{, }\AttributeTok{lwd =} \DecValTok{2}\NormalTok{)}
\CommentTok{\# temp is very slowly increasing.}
\CommentTok{\# \textasciitilde{}\textasciitilde{}\textasciitilde{}\textasciitilde{}\textasciitilde{}\textasciitilde{}\textasciitilde{}\textasciitilde{}\textasciitilde{}\textasciitilde{}\textasciitilde{}\textasciitilde{}\textasciitilde{}\textasciitilde{}\textasciitilde{}\textasciitilde{}\textasciitilde{}\textasciitilde{}\textasciitilde{}\textasciitilde{}\textasciitilde{}\textasciitilde{}\textasciitilde{}\textasciitilde{}\textasciitilde{}\textasciitilde{}\textasciitilde{}\textasciitilde{}\textasciitilde{}\textasciitilde{}\textasciitilde{}\textasciitilde{}\textasciitilde{}\textasciitilde{}\textasciitilde{}\textasciitilde{}\textasciitilde{}\textasciitilde{}\textasciitilde{}\textasciitilde{}\textasciitilde{}\textasciitilde{}\textasciitilde{}\textasciitilde{}\textasciitilde{}\textasciitilde{}\textasciitilde{}\textasciitilde{}\textasciitilde{}\textasciitilde{}\textasciitilde{}\textasciitilde{}\textasciitilde{}\textasciitilde{}\textasciitilde{}\textasciitilde{}\textasciitilde{}\textasciitilde{}\textasciitilde{}\textasciitilde{}\textasciitilde{}\textasciitilde{}\textasciitilde{}\textasciitilde{}\textasciitilde{}\textasciitilde{}\textasciitilde{}\textasciitilde{}\textasciitilde{}\textasciitilde{}\textasciitilde{}\textasciitilde{}\textasciitilde{}\textasciitilde{}\textasciitilde{}\textasciitilde{}\textasciitilde{}}
\CommentTok{\# square the time}
\NormalTok{time2 }\OtherTok{\textless{}{-}}\NormalTok{ (time}\SpecialCharTok{\^{}}\DecValTok{2}\NormalTok{)}
\CommentTok{\# quadratic trend}
\NormalTok{quadratic.fun }\OtherTok{\textless{}{-}} \ControlFlowTok{function}\NormalTok{(timeseries)\{}
  \CommentTok{\# create quadratic fit model}
\NormalTok{  quadratic.fit }\OtherTok{\textless{}\textless{}{-}} \FunctionTok{lm}\NormalTok{(timeseries }\SpecialCharTok{\textasciitilde{}}\NormalTok{ time }\SpecialCharTok{+}\NormalTok{ time2)}
  \CommentTok{\# create quadratic fitted}
\NormalTok{  quadratic.fit }\SpecialCharTok{\%\textgreater{}\%} \FunctionTok{fitted}\NormalTok{() }\SpecialCharTok{\%\textgreater{}\%} \FunctionTok{ts}\NormalTok{(}\AttributeTok{start =} \FunctionTok{c}\NormalTok{(}\DecValTok{1884}\NormalTok{,}\DecValTok{12}\NormalTok{), }\AttributeTok{end =} \FunctionTok{c}\NormalTok{(}\DecValTok{2019}\NormalTok{,}\DecValTok{12}\NormalTok{), }\AttributeTok{frequency =} \DecValTok{12}\NormalTok{) }\OtherTok{{-}\textgreater{}\textgreater{}}
\NormalTok{    quadratic.fitted}
\NormalTok{\}}
\CommentTok{\# run quad function}
\FunctionTok{quadratic.fun}\NormalTok{(slicedWindow)}
\CommentTok{\# check summary}
\FunctionTok{summary}\NormalTok{(quadratic.fit)}
\end{Highlighting}
\end{Shaded}

\begin{verbatim}
## 
## Call:
## lm(formula = timeseries ~ time + time2)
## 
## Residuals:
##      Min       1Q   Median       3Q      Max 
## -13.8660  -4.8178  -0.1242   5.0853  11.9001 
## 
## Coefficients:
##             Estimate Std. Error t value Pr(>|t|)    
## (Intercept)   13.441      0.410  32.780   <2e-16 ***
## time          -0.605      1.894  -0.319    0.749    
## time2          2.018      1.834   1.101    0.271    
## ---
## Signif. codes:  0 '***' 0.001 '**' 0.01 '*' 0.05 '.' 0.1 ' ' 1
## 
## Residual standard error: 5.51 on 1618 degrees of freedom
## Multiple R-squared:  0.006208,   Adjusted R-squared:  0.00498 
## F-statistic: 5.054 on 2 and 1618 DF,  p-value: 0.006485
\end{verbatim}

\begin{Shaded}
\begin{Highlighting}[]
\CommentTok{\# add linear fitted lines}
\FunctionTok{lines}\NormalTok{(quadratic.fitted, }\AttributeTok{col =} \StringTok{"green"}\NormalTok{, }\AttributeTok{lwd =} \DecValTok{2}\NormalTok{)}
\CommentTok{\# \textasciitilde{}\textasciitilde{}\textasciitilde{}\textasciitilde{}\textasciitilde{}\textasciitilde{}\textasciitilde{}\textasciitilde{}\textasciitilde{}\textasciitilde{}\textasciitilde{}\textasciitilde{}\textasciitilde{}\textasciitilde{}\textasciitilde{}\textasciitilde{}\textasciitilde{}\textasciitilde{}\textasciitilde{}\textasciitilde{}\textasciitilde{}\textasciitilde{}\textasciitilde{}\textasciitilde{}\textasciitilde{}\textasciitilde{}\textasciitilde{}\textasciitilde{}\textasciitilde{}\textasciitilde{}\textasciitilde{}\textasciitilde{}\textasciitilde{}\textasciitilde{}\textasciitilde{}\textasciitilde{}\textasciitilde{}\textasciitilde{}\textasciitilde{}\textasciitilde{}\textasciitilde{}\textasciitilde{}\textasciitilde{}\textasciitilde{}\textasciitilde{}\textasciitilde{}\textasciitilde{}\textasciitilde{}\textasciitilde{}\textasciitilde{}\textasciitilde{}\textasciitilde{}\textasciitilde{}\textasciitilde{}\textasciitilde{}\textasciitilde{}\textasciitilde{}\textasciitilde{}\textasciitilde{}\textasciitilde{}\textasciitilde{}\textasciitilde{}\textasciitilde{}\textasciitilde{}\textasciitilde{}\textasciitilde{}\textasciitilde{}\textasciitilde{}\textasciitilde{}\textasciitilde{}\textasciitilde{}\textasciitilde{}\textasciitilde{}\textasciitilde{}\textasciitilde{}\textasciitilde{}\textasciitilde{}}
\CommentTok{\# cube time}
\NormalTok{time3 }\OtherTok{\textless{}{-}}\NormalTok{ time}\SpecialCharTok{\^{}}\DecValTok{3}
\CommentTok{\# cubic function}
\NormalTok{cube.fun }\OtherTok{\textless{}{-}} \ControlFlowTok{function}\NormalTok{(timeseries)\{}
  \CommentTok{\# create cubic fit model}
\NormalTok{  cubic.fit }\OtherTok{\textless{}\textless{}{-}} \FunctionTok{lm}\NormalTok{(timeseries }\SpecialCharTok{\textasciitilde{}}\NormalTok{ time }\SpecialCharTok{+}\NormalTok{ time2 }\SpecialCharTok{+}\NormalTok{ time3)}
  \CommentTok{\# create cubic fitted}
\NormalTok{  cubic.fit }\SpecialCharTok{\%\textgreater{}\%} \FunctionTok{fitted}\NormalTok{() }\SpecialCharTok{\%\textgreater{}\%} \FunctionTok{ts}\NormalTok{(}\AttributeTok{start =} \FunctionTok{c}\NormalTok{(}\DecValTok{1884}\NormalTok{,}\DecValTok{12}\NormalTok{), }\AttributeTok{end =} \FunctionTok{c}\NormalTok{(}\DecValTok{2019}\NormalTok{,}\DecValTok{12}\NormalTok{), }\AttributeTok{frequency =} \DecValTok{12}\NormalTok{) }\OtherTok{{-}\textgreater{}\textgreater{}} 
\NormalTok{    quadratic.fitted}
\NormalTok{\}}
\CommentTok{\# run cubic function}
\FunctionTok{cube.fun}\NormalTok{(slicedWindow)}
\CommentTok{\# check summary}
\FunctionTok{summary}\NormalTok{(cubic.fit)}
\end{Highlighting}
\end{Shaded}

\begin{verbatim}
## 
## Call:
## lm(formula = timeseries ~ time + time2 + time3)
## 
## Residuals:
##      Min       1Q   Median       3Q      Max 
## -13.7276  -4.8674  -0.2116   5.0442  11.8990 
## 
## Coefficients:
##             Estimate Std. Error t value Pr(>|t|)    
## (Intercept)  12.8291     0.5458  23.505   <2e-16 ***
## time          6.7467     4.7283   1.427   0.1538    
## time2       -16.3667    10.9894  -1.489   0.1366    
## time3        12.2565     7.2237   1.697   0.0899 .  
## ---
## Signif. codes:  0 '***' 0.001 '**' 0.01 '*' 0.05 '.' 0.1 ' ' 1
## 
## Residual standard error: 5.506 on 1617 degrees of freedom
## Multiple R-squared:  0.007975,   Adjusted R-squared:  0.006134 
## F-statistic: 4.333 on 3 and 1617 DF,  p-value: 0.004744
\end{verbatim}

\begin{Shaded}
\begin{Highlighting}[]
\CommentTok{\# add cubic line}
\FunctionTok{lines}\NormalTok{(time, }
\NormalTok{      cubic.fit }\SpecialCharTok{\%\textgreater{}\%} \FunctionTok{fitted}\NormalTok{(),}
      \AttributeTok{col =} \StringTok{\textquotesingle{}yellow\textquotesingle{}}\NormalTok{,}
      \AttributeTok{lwd =} \DecValTok{3}\NormalTok{)}
\end{Highlighting}
\end{Shaded}

\includegraphics{testing_files/figure-latex/unnamed-chunk-9-1.pdf}

\begin{Shaded}
\begin{Highlighting}[]
\CommentTok{\# check fit}
\FunctionTok{AIC}\NormalTok{(linear.fit) }
\end{Highlighting}
\end{Shaded}

\begin{verbatim}
## [1] 10136.89
\end{verbatim}

\begin{Shaded}
\begin{Highlighting}[]
\FunctionTok{AIC}\NormalTok{(quadratic.fit)}
\end{Highlighting}
\end{Shaded}

\begin{verbatim}
## [1] 10137.68
\end{verbatim}

\begin{Shaded}
\begin{Highlighting}[]
\FunctionTok{AIC}\NormalTok{(cubic.fit)}
\end{Highlighting}
\end{Shaded}

\begin{verbatim}
## [1] 10136.8
\end{verbatim}

The section above looks to create a linear, quadratic and cubic model
for the selected sliced dataset and plots their values with a line.

\begin{Shaded}
\begin{Highlighting}[]
\CommentTok{\# 4.2 {-} Select Trend}
\CommentTok{\# Select a trend model for each time series using an appropriate criteria. }
\CommentTok{\# Are the models selected all the same? If not is there a pattern depending on }
\CommentTok{\# the region and/or the group (max, mean and min)?}

\CommentTok{\# All datasets had a lower AIC result for linear fit,}
\CommentTok{\# except for TMAX{-}England\_SE\_and\_Central\_S}
\end{Highlighting}
\end{Shaded}

Going through the trend for all of the datasets manually took a very
long time (which I have omitted), but for most of them the AIC for
linear.fit seemed to be the best model.

\begin{Shaded}
\begin{Highlighting}[]
\CommentTok{\# 4.3 {-} Estimate Seasonality}
\CommentTok{\# After removing the trend using the model selected in the previous step, }
\CommentTok{\# use the output to estimate the seasonality of each time series employing averaging and sine{-}cosine models. }
\CommentTok{\# Compare your results and use appropriate plots and/or tables to confirm your observations.}

\NormalTok{get.seasonality }\OtherTok{\textless{}{-}} \ControlFlowTok{function}\NormalTok{(timeseries)\{}
  \CommentTok{\# get the residuals}
\NormalTok{  sw.notrend }\OtherTok{\textless{}\textless{}{-}}\NormalTok{ (timeseries }\SpecialCharTok{{-}} \FunctionTok{fitted}\NormalTok{(linear.fit))}
  \CommentTok{\# get seasonal means}
  \FunctionTok{tapply}\NormalTok{(sw.notrend, }\FunctionTok{cycle}\NormalTok{(sw.notrend), mean)}
  \CommentTok{\# create months variable as factor}
\NormalTok{  months }\OtherTok{\textless{}{-}}\NormalTok{ sw.notrend }\SpecialCharTok{\%\textgreater{}\%} \FunctionTok{cycle}\NormalTok{() }\SpecialCharTok{\%\textgreater{}\%} \FunctionTok{as.factor}\NormalTok{()}
  \CommentTok{\# seasonal means}
\NormalTok{  sliced.seas }\OtherTok{\textless{}\textless{}{-}} \FunctionTok{lm}\NormalTok{(sw.notrend }\SpecialCharTok{\textasciitilde{}}\NormalTok{ months }\SpecialCharTok{{-}} \DecValTok{1}\NormalTok{)}
    \CommentTok{\# evaluate harmonic seasonality}
  \CommentTok{\# create an empty matrix}
\NormalTok{  SIN }\OtherTok{\textless{}\textless{}{-}}\NormalTok{ COS }\OtherTok{\textless{}\textless{}{-}}  \FunctionTok{matrix}\NormalTok{(}\AttributeTok{nrow =} \FunctionTok{length}\NormalTok{(time), }\AttributeTok{ncol =} \DecValTok{6}\NormalTok{)}\CommentTok{\# 6 = freq/2}
  \CommentTok{\# loop through}
  \ControlFlowTok{for}\NormalTok{(i }\ControlFlowTok{in} \DecValTok{1}\SpecialCharTok{:}\DecValTok{6}\NormalTok{)\{}
\NormalTok{    SIN[,i] }\OtherTok{\textless{}{-}} \FunctionTok{sin}\NormalTok{(}\DecValTok{2}\SpecialCharTok{*}\NormalTok{pi}\SpecialCharTok{*}\NormalTok{i}\SpecialCharTok{*}\NormalTok{time)}
\NormalTok{    COS[,i] }\OtherTok{\textless{}{-}} \FunctionTok{cos}\NormalTok{(}\DecValTok{2}\SpecialCharTok{*}\NormalTok{pi}\SpecialCharTok{*}\NormalTok{i}\SpecialCharTok{*}\NormalTok{time)}
\NormalTok{  \}}
  \CommentTok{\# model all season harmonic}
  \CommentTok{\# model notrend against all values with {-}1}
\NormalTok{  slice.har1 }\OtherTok{\textless{}\textless{}{-}} \FunctionTok{lm}\NormalTok{(sw.notrend }\SpecialCharTok{\textasciitilde{}}\NormalTok{ . }\SpecialCharTok{{-}}\DecValTok{1}\NormalTok{ ,}
                   \FunctionTok{data.frame}\NormalTok{(}\AttributeTok{SIN =}\NormalTok{ SIN[,}\DecValTok{1}\NormalTok{], }\AttributeTok{COS =}\NormalTok{ COS[,}\DecValTok{1}\NormalTok{]))}
  \CommentTok{\# slice 2}
\NormalTok{  slice.har2 }\OtherTok{\textless{}\textless{}{-}} \FunctionTok{lm}\NormalTok{(sw.notrend }\SpecialCharTok{\textasciitilde{}}\NormalTok{ . }\SpecialCharTok{{-}}\DecValTok{1}\NormalTok{ ,}
                   \FunctionTok{data.frame}\NormalTok{(}\AttributeTok{SIN =}\NormalTok{ SIN[,}\DecValTok{1}\SpecialCharTok{:}\DecValTok{2}\NormalTok{], }\AttributeTok{COS =}\NormalTok{ COS[,}\DecValTok{1}\SpecialCharTok{:}\DecValTok{2}\NormalTok{]))}
  \CommentTok{\# slice 3}
\NormalTok{  slice.har3 }\OtherTok{\textless{}\textless{}{-}} \FunctionTok{lm}\NormalTok{(sw.notrend }\SpecialCharTok{\textasciitilde{}}\NormalTok{ . }\SpecialCharTok{{-}}\DecValTok{1}\NormalTok{ ,}
                    \FunctionTok{data.frame}\NormalTok{(}\AttributeTok{SIN =}\NormalTok{ SIN[,}\DecValTok{1}\SpecialCharTok{:}\DecValTok{3}\NormalTok{], }\AttributeTok{COS =}\NormalTok{ COS[,}\DecValTok{1}\SpecialCharTok{:}\DecValTok{3}\NormalTok{]))}
  \CommentTok{\# slice 4}
\NormalTok{  slice.har4 }\OtherTok{\textless{}\textless{}{-}} \FunctionTok{lm}\NormalTok{(sw.notrend }\SpecialCharTok{\textasciitilde{}}\NormalTok{ . }\SpecialCharTok{{-}}\DecValTok{1}\NormalTok{ ,}
                    \FunctionTok{data.frame}\NormalTok{(}\AttributeTok{SIN =}\NormalTok{ SIN[,}\DecValTok{1}\SpecialCharTok{:}\DecValTok{4}\NormalTok{], }\AttributeTok{COS =}\NormalTok{ COS[,}\DecValTok{1}\SpecialCharTok{:}\DecValTok{4}\NormalTok{]))}
  \CommentTok{\# slice 5}
\NormalTok{  slice.har5 }\OtherTok{\textless{}\textless{}{-}} \FunctionTok{lm}\NormalTok{(sw.notrend }\SpecialCharTok{\textasciitilde{}}\NormalTok{ . }\SpecialCharTok{{-}}\DecValTok{1}\NormalTok{ ,}
                    \FunctionTok{data.frame}\NormalTok{(}\AttributeTok{SIN =}\NormalTok{ SIN[,}\DecValTok{1}\SpecialCharTok{:}\DecValTok{5}\NormalTok{], }\AttributeTok{COS =}\NormalTok{ COS[,}\DecValTok{1}\SpecialCharTok{:}\DecValTok{5}\NormalTok{]))}
  \CommentTok{\# slice 6}
\NormalTok{  slice.har6 }\OtherTok{\textless{}\textless{}{-}} \FunctionTok{lm}\NormalTok{(sw.notrend }\SpecialCharTok{\textasciitilde{}}\NormalTok{ . }\SpecialCharTok{{-}}\DecValTok{1}\NormalTok{ ,}
                    \FunctionTok{data.frame}\NormalTok{(}\AttributeTok{SIN =}\NormalTok{ SIN[,}\DecValTok{1}\SpecialCharTok{:}\DecValTok{6}\NormalTok{], }\AttributeTok{COS =}\NormalTok{ COS[,}\DecValTok{1}\SpecialCharTok{:}\DecValTok{6}\NormalTok{]))}
\NormalTok{\}}
\CommentTok{\# run seasonality fun}
\FunctionTok{get.seasonality}\NormalTok{(slicedWindow)}

\NormalTok{getAIC }\OtherTok{\textless{}{-}} \FunctionTok{data.frame}\NormalTok{(}
  \AttributeTok{slice.har.1 =} \FunctionTok{AIC}\NormalTok{(slice.har1),}
  \AttributeTok{slice.har.2 =} \FunctionTok{AIC}\NormalTok{(slice.har2),}
  \AttributeTok{slice.har.3 =} \FunctionTok{AIC}\NormalTok{(slice.har3),}
  \AttributeTok{slice.har.4 =} \FunctionTok{AIC}\NormalTok{(slice.har4),}
  \AttributeTok{slice.har.5 =} \FunctionTok{AIC}\NormalTok{(slice.har5),}
  \AttributeTok{slice.har.6 =} \FunctionTok{AIC}\NormalTok{(slice.har6)}
\NormalTok{)}
\CommentTok{\# print with knitr table}
\FunctionTok{kable}\NormalTok{(getAIC, }\AttributeTok{caption =} \StringTok{"AIC\textquotesingle{}s for all harmonic seasonalities"}\NormalTok{)}
\end{Highlighting}
\end{Shaded}

\begin{longtable}[]{@{}rrrrrr@{}}
\caption{AIC's for all harmonic seasonalities}\tabularnewline
\toprule
slice.har.1 & slice.har.2 & slice.har.3 & slice.har.4 & slice.har.5 &
slice.har.6 \\ \addlinespace
\midrule
\endfirsthead
\toprule
slice.har.1 & slice.har.2 & slice.har.3 & slice.har.4 & slice.har.5 &
slice.har.6 \\ \addlinespace
\midrule
\endhead
10134.41 & 10137.16 & 10140.49 & 10144.25 & 10147.99 &
10151.88 \\ \addlinespace
\bottomrule
\end{longtable}

\begin{Shaded}
\begin{Highlighting}[]
\CommentTok{\# sort decreasing}
\NormalTok{getAIC }\SpecialCharTok{\%\textgreater{}\%} \FunctionTok{sort}\NormalTok{(}\AttributeTok{decreasing =}\NormalTok{ F)}
\end{Highlighting}
\end{Shaded}

\begin{verbatim}
##   slice.har.1 slice.har.2 slice.har.3 slice.har.4 slice.har.5 slice.har.6
## 1    10134.41    10137.16    10140.49    10144.25    10147.99    10151.88
\end{verbatim}

\begin{Shaded}
\begin{Highlighting}[]
\CommentTok{\#summary(slice.har1)}
\CommentTok{\#summary(slice.har2)}
\CommentTok{\#summary(slice.har3)}
\CommentTok{\#summary(slice.har4)}
\CommentTok{\#summary(slice.har5)}
\CommentTok{\#summary(slice.har6)}

\CommentTok{\# removed as it doesnt work.}
\CommentTok{\# plot(slicedWindow,}
\CommentTok{\#      main = "AVG TEMPS IN UK",}
\CommentTok{\#      xlab = "Year",}
\CommentTok{\#      ylab = "AVG TEMP",}
\CommentTok{\#      type = "l")}
\CommentTok{\# lines doesnt work for some reason.}
\CommentTok{\# lines(time, }
\CommentTok{\#       fitted(slice.har2),}
\CommentTok{\#       lwd = 3,}
\CommentTok{\#       type = "l",}
\CommentTok{\#       col = "blue")}
\end{Highlighting}
\end{Shaded}

This section looked to estimate the seasonality for the sliced dataset
whilst looking at the AIC results for each harmonic seasonality.

\begin{Shaded}
\begin{Highlighting}[]
\CommentTok{\# 4.4 {-} Select Seasonality}
\CommentTok{\# Select a seasonal model for each time series using an appropriate criteria.}
\CommentTok{\# Are the models selected all the same? }
\CommentTok{\# If not is there a pattern depending on the region and/or the group (max, min and mean)?}


\CommentTok{\# 4.5 {-} Estimate Seasonality}
\CommentTok{\# Estimate a combined model for trend and seasonality using the results of the previous steps.}
\CommentTok{\# Call this model “final”.}




\CommentTok{\# 4.6 {-} Estimate Seasonality}
\CommentTok{\# Estimate trend and seasonality using a combined quadratic and sin{-}cosine (of order 2) models.}
\CommentTok{\# Call this model “test”}
\end{Highlighting}
\end{Shaded}

\hypertarget{task-5}{%
\section{5 - Task 5}\label{task-5}}

\hypertarget{reflection}{%
\section{6 - Reflection}\label{reflection}}

There were many parts within this coursework which I really didn't
understand so a lot of this was left out - I am hopeful that the small
parts I did were enough to get me through.

\end{document}
