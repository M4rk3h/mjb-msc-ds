
%%%%%%%%%%%%%%%%%%%%%%%%%%%%%%%%%%%%%%%%%%%%%%%%%%%%%%%%%%%%%%%%%%%%%%%%%%%%%%%%%%%%%%%%%%%%%%%%%%%%%%%%%%%
% These are some useful packages that make additional commands available to the Latex file:
%
% amsmath, amsthm, amssymb - standard maths packages - needed for ANY serious mathematical writing
% url - needed for inputting urls so that they format properly
% graphix - for pictures - any inputting of picture files you'll need this
% float - allows you to use [H}] next to figures and tables to keep them PERFECTLY in place
% multirow and multicol do what they say on the tin
% subfigure - for laying out subfigures
% lscape - to put pages in landscape if you have pictures that you want on a landscape page
% array - additional array features
% ltxtable - so that you can format columns
% color - does what it says on the tin
% colortbl - color for tables
% appendix - appendix
% rotating - The package rotating gives you the possibility to rotate any object of an arbitrary angle. 
% setspace - used for double or 1.5 spacing
% hyperref - This package creates links in your pdf, allowing easy navigation of your report
%%%%%%%%%%%%%%%%%%%%%%%%%%%%%%%%%%%%%%%%%%%%%%%%%%%%%%%%%%%%%%%%%%%%%%%%%%%%%%%%%%%%%%%%%%%%%%%%%%%%%%%%%%
\usepackage{amsmath,amsthm,amssymb,url,graphicx,float,multicol,multirow,subfigure,lscape,array,ltxtable,colortbl,color,appendix,rotating,setspace,hyperref}

% The geometry package allows you to set up the margin space for your pages
\usepackage[right=2.5cm, left=2.5cm, top=2.5cm, bottom=2.5cm]{geometry}

% the caption package is used for captions, to ensure your captions are always centered
\usepackage[center]{caption}

%%%%%%%%%%%%%%%%%%%%%%%%%%%%%%%%%%%%%%%%%%%%%%%%%%%%%%%%%%%%%%%%%%%%%%%%%%%%%%%%%%%%%%%%%%%%%%%%%%%%%%%%%%
% this is so that you can have subsubsubsubsubsectioins if you want
\stepcounter{secnumdepth}
\stepcounter{tocdepth}
%%%%%%%%%%%%%%%%%%%%%%%%%%%%%%%%%%%%%%%%%%%%%%%%%%%%%%%%%%%%%%%%%%%%%%%%%%%%%%%%%%%%%%%%%%%%%%%%%%%%%%%%%%

%%%%%%%%%%%%%%%%%%%%%%%%%%%%%%%%%%%%%%%%%%%%%%%%%%%%%%%%%%%%%%%%%%%%%%%%%%%%%%%%%%%%%%%%%%%%%%%%%%%%%%%%%%
% Command for creating the dissertation title page in accordance with university requirements
\newcommand{\thesistitle}[5]{
\thispagestyle{empty}
\begin{center}

\renewcommand{\baselinestretch}{0.7}%
\textbf{\huge #1}
%\renewcommand{\baselinestretch}{\onehalfspacing}%
\doublespacing

\vspace{1.0cm}

\Large #2

\large #3

\vspace{\fill}

\vspace{\fill}

A submission presented in  \\
partial fulfilment of the requirements \\
of the University of South Wales/Prifysgol De Cymru \\
for the degree of #4 \\

\vspace{\fill}

#5 \\
\end{center}
}

\usepackage{natbib}
\bibliographystyle{unsrtnat}
%%%%%%%%%%%%%%%%%%%%%%%%%%%%%%%%%%%%%%%%%%%%%%%%%%%%%%%%%%%%%%%%%%%%%%%%%%%%%%%%%%%%%%%%%%%%%%%%%%%%%%%%%%

